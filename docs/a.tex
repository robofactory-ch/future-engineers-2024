% !TeX program = xelatex
\documentclass[a4paper]{scrarticle}

\usepackage[ngerman]{babel}
\usepackage[utf8]{inputenc}
\usepackage[T1]{fontenc}

\usepackage[left=2.25cm, right=2.25cm, top=3.00cm, bottom=3.50cm]{geometry}
\usepackage[headsepline, footsepline]{scrlayer-scrpage}
\renewcommand*{\headfont}{\normalfont}
\usepackage{csquotes}

\usepackage{graphicx}
\usepackage[figurename=Abb.]{caption}

\usepackage{amsmath}
\usepackage{amssymb}

\usepackage{tabto}
\usepackage{xcolor}
\usepackage{enumitem}
\usepackage{blindtext, showframe}
\usepackage{fontspec}
\usepackage{lipsum} % for dummy text

\definecolor{AKSAcolor}{rgb}{0.64,0.44,0.32}
\newfontfamily\AKAfont{AKA}

% Redefine sectioning commands to set font
\makeatletter
\renewcommand\section{\@startsection {section}{1}{\z@}%
                                   {-3.5ex \@plus -1ex \@minus -.2ex}%
                                   {2.3ex \@plus.2ex}%
                                   {\Huge\AKAfont}}
\renewcommand\subsection{\@startsection{subsection}{2}{\z@}%
                                     {-3.25ex\@plus -1ex \@minus -.2ex}%
                                     {1.5ex \@plus .2ex}%
                                     {\Large\AKAfont}}
\renewcommand{\maketitle}{%
																		 \begin{titlepage}
																			 \null\vfill % Add space at the top
																			 \begin{center}
																				 {\huge\@title\par}%
																				 \vspace{0.5cm} % Adjust spacing between title and author
																				 {\large\@subtitle\par} % Add the subtitle
																				 \vspace{1.5cm} % Adjust spacing between author and subtitle
																				 {\Large\@author\par}
																				 \vspace{0.5cm} % Adjust spacing between subtitle and date
																				 {\large\@date\par} % Add the date
																			 \end{center}
																			 \vfill % Fill remaining space at the bottom
																			 \begin{center}
																				Eingereicht bei der WRO Regio in Waldkirch bei Freiburg
																			 \end{center}
																			 \@thanks % If you have any thanks or notes, they will be printed here
																		 \end{titlepage}%
																	 }

\makeatother

\usepackage{hyperref}
\hypersetup{colorlinks=false, pdfborder={0 0 0}, pdftitle=}



\begin{document}

\title{\AKAfont\Huge\textcolor{AKSAcolor}{Dokumentation}}
\subtitle{Future Engineers 2024}
\author{robofactory (Jesse Born, Julian von Hoff)}
\date{April 2024}

\pagenumbering{gobble} % Suppress page numbering
\ihead{Future Engineers Dokumentation}
\ifoot{\\robofactory, 2024}

\maketitle
\clearpage
\newpage


\pagenumbering{arabic}

\section{Motorisierung}

Als Hauptantrieb verwenden wir eine Handelsübliche Kombination aus Fahrtenregler und Motor, die aus dem Modelbau stammt.
Wir verwenden einen 21.5T gewickelten Motor von Tamiya mit eingebautem Drehgeber und das dazugehörige ESC TBLE-04SR. Über ein Differentialgetriebe und eine gesamthafte Untersetzung um den Faktor 3.122 bringt unser Fahrzeug seine Motorleistung über einen Heckantrieb auf den Boden.


\section{Energie \& Sensoren}

Der einzige Sensor den wir verbaut haben, ist eine RGB-D Kamera von Orbbec, eine Astra Embedded S.
Dieses relativ preiswerte Auslaufprodukt ermöglicht nebst der klassischen Webcamfunktion auch das erfassen von Tiefeninformation mittels Struckturiertem Infrarotlicht.
Via USB / UVC-Protokoll wird das Kamerabild an unseren Hauptrechner, einen Raspberry Pi 4B übertragen.

\section{Hindernisse}

Das von unserer Kamera erfasst RGB- bzw. BGR-Bild folgt nun 2 Pfaden: Der erste ist dafür zuständig, während dem Hindernissrennen die farbigen Klötze zu erkennen. Entlang dem zweiten Pfad werden mittels Graustufenbild und Hough-Lines-Algorithmus die Banden des Spielfeldes erkannt und lokalisiert.

\subsection{Banden}

Als erstes erstellen wir ein Bild, in dem nur alle (fast) schwarzen Pixel abgebildet sind. Darin suchen wir anschliessend nach kontrastreichen Kanten.
Haben wir die Kanten gefunden, segmentieren wir mittels probabilistischer Hough-Transformation zusamenhängende Kanten aus.
Nun filtern wir diese Kanten weiter, Wir überspringen alle Linien, welche...
\begin{enumerate}
	\item {keinen Endpunkt in einem Streifen um die Bildmitte haben}
	\item {eine Nulllänge haben}
	\item {komplett über der Bildmitte liegen}
	\item {vertikal im Bild stehen}
\end{enumerate}
So erhalten wir pro Wand eine Linie, an der unteren Kante dieser, an der Stelle wo sie auf die Spielfeldmatte trifft.
Mittels des 2. Strahlensatzes können wir nun den Abstand schätzen.
$$
d = \frac{1}{(y_{Endpunkt}-y_{Bildmitte})} * s_d
$$
($d$: Abstand, $y_{Endpunkt}$: y-Koordinate eines Linienendpunktes im Bild, $y_{Bildmitte}$: Parallaxenmitte des Kamerbildes und $s_d$: konstanter Faktor, der sich zwar aus den intrinsischen Kameraparametern ableitet, jedoch experimentel bestummen wurde. )
\clearpage
Wände werden nun in drei Klassen unterteilt:
\begin{itemize}
	\item Rechts: Eine Wand am rechten Bildrand
	\item Links: Eine Wand am linken Bildrand
	\item Mitte: Eine Wand, deren Steigungswinkel unter $0.05$ (rad) liegt.
\end{itemize}
Diese Klassifizierung wird auch zu Beginn des Laufs zur feststellung der Rundenrichtung verwendet – ist eine linke Wand sichtbar, wird im Uhrzeigersinn gefahren.

\subsection{Hindernisse}

Das originale RGB-Bild wird in den HSV-Farbraum umgerechnet, um rot und grün deutlich unterscheiden zu können. Diesen Trick kennen und nutzten wir bereits seit mehreren Jahren in der Kategorie RoboMission der WRO.
Pixel die nun zwischen der Rot- bzw. Grüngrenzen liegen, werden als der Farbe entsprechend gespeichert. 
Auf diesen Binärbildern werden nun die Konturen erkannt und die Mittelpunkte sowie die Breite und Höhe bestummen.

Analog zu oben, jedoch nur mit der Höhe der erkannten Kontur, wird mittels Strahlensazt der Abstand zum Hinderniss geschätzt.

\subsection{Steuersignal}

\[ \frac{\sum_{}\frac{1}{d_i}}{T} \]


\section{Fotos}
\section{Engineering \& Design}
Der Bausatz des Chassis stammt ebenfalls aus dem Hause Tamiya – den klassiker M-08 Concept haben wir modifiziert, dass der Einschlag der Lenkung grösser und die Federung so steif wie nur möglich ist. (Siehe Kapitel "3 - Hindernisse"). 
Die vordere Stossstange, bestehend aus einem dichten Schaumstoff haben wir an der Bandsäge gekürzt, so dass die gesamtlänge des Autos auch in die vorgegebenen 300 mm passt.
Ergänzt haben wir den fahrbaren Unterbau mit einer Kamerahalterung über der vorderen Stossstange. Die Kamera hält darin Reibungsbasiert und lässt sich für Wartungsarbeiten rund um diese leicht entfernen.
Aus Notdurft – um unseren Wanderkennungsalgorithmus zu unterstützen – haben wir die hydraulische Federung des Chassisbausatzes mit soliden, 3d-gedruckten Teilen ersetzt. Damit vermeiden wir, das unser Auto rollt oder nickt – denn dies macht die visuelle Distanzschätzung ungenau bis unbrauchbar.

\section{Anhänge}


\end{document}